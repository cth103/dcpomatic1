\documentclass{article}
\usepackage{amsmath}
\begin{document}

Here is what resampling we need to do.  Content video is at $C_V$ fps, audio at $C_A$.  

\section{Easy case 1}

$C_V$ and $C_A$ are both DCI rates, e.g.\ if $C_V = 24$, $C_A = 48\times{}10^3$.

\medskip
\textbf{Nothing to do.}

\section{Easy case 2}

$C_V$ is a DCI rate, $C_A$ is not.  e.g.\ if $C_V = 24$, $C_A = 44.1\times{}10^3$.

\medskip
\textbf{Resample $C_A$ to the DCI rate.}

\section{Hard case 1}
\label{sec:hard1}

$C_V$ is not a DCI rate, $C_A$ is, e.g.\ if $C_V = 25$, $C_A =
48\times{}10^3$.  We will run the video at a nearby DCI rate $F_V$,
meaning that it will run faster or slower than it should.  We resample
the audio to $C_V C_A / F_V$ and mark it as $C_A$ so that it, too,
runs faster or slower by the corresponding factor.

e.g.\ if $C_V = 25$, $F_V = 24$ and $C_A = 48\times{}10^3$, we
resample audio to $25 * 48\times{}10^3 / 24 = 50\times{}10^3$.

\medskip
\textbf{Resample $C_A$ to $C_V C_A / F_V$}

\section{Hard case 2}

Neither $C_V$ nor $C_A$ is not a DCI rate, e.g.\ if $C_V = 25$, $C_A =
44.1\times{}10^3$.  We will run the video at a nearby DCI rate $F_V$,
meaning that it will run faster or slower than it should.  We first
resample the audio to a DCI rate $F_A$, then perform as with
Section~\ref{sec:hard1} above.

\medskip
\textbf{Resample $C_A$ to $C_V F_A / F_V$}


\section{The general case}

Given a DCP running at $F_V$ and $F_A$ and a piece of content at $C_V$
and $C_A$, resample the audio to $R_A$ where
\begin{align*}
R_A &= \frac{C_V F_A}{F_V}
\end{align*}

\end{document}
